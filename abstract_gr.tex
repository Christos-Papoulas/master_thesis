\thispagestyle{empty}
\selectlanguage{greek}
\begin{titlepage}
\begin{center}
{\bf\Large{Περίληψη}}\\
\end{center}

\indent  

Μια νέα τάση που γεφυρώνει την ανάπτυξη με την λειτουργία των κατανεμημένων εφαρμογών, γνωστή ως \selectlanguage{english}DevOps\selectlanguage{greek}, έχει γνωρίσει σημαντική ανάπτυξη τα τελευταία χρόνια. Μηχανικοί \selectlanguage{english}DevOps\selectlanguage{greek} με εξειδίκευση στην προσαρμογή και εγκατάσταση εφαρμογών σε περιβάλλοντα υπολογιστικού νέφους (γνωστοί και ως \selectlanguage{english}cloud deployment specialists\selectlanguage{greek}), χρησιμοποιούν όλο και περισσότερο συγκεκριμένα εργαλεία για την εγκατάσταση και λειτουργία των εφαρμογών τους, όπως το \selectlanguage{english}Chef Supermarket\selectlanguage{greek} και το \selectlanguage{english}IBM Bluemix\selectlanguage{greek}. Παρά την σημαντική αυτοματοποίηση που προσφέρουν αυτά τα εργαλεία, η εύρεση των χαρακτηριστικών μιας επιτυχούς εγκατάστασης κατά περίπτωση απαιτεί συζήτηση με ειδικούς, συχνά σε διαδικτυακά τεχνικά φόρουμ και κοινωνικά δίκτυα.

Εντός της κοινότητας των μηχανικών \selectlanguage{english}DevOps\selectlanguage{greek}, η συζήτηση γύρω από την δομή των εφαρμογών και τα αποτελέσματα των διαφόρων παραμέτρων εγκατάστασης τους σε υπολογιστικά νέφη μπορεί να γίνει πιο εποικοδομητική αν οι τεχνολογίες κοινωνικής δικτύωσης εμπλουτιστούν με την γνώση που υπάρχει σε αποθετήρια δεδομένων χρηστών της κοινότητας \selectlanguage{english}DevOps\selectlanguage{greek} (όπως π.χ. το \selectlanguage{english}Chef Supermarket\selectlanguage{greek}). Σε αυτήν την εργασία προτείνουμε μια πλατφόρμα κοινωνικής δικτύωσης (που παίρνει το όνομά της από το \selectlanguage{english}PaaSage EU project\selectlanguage{greek}) όπου οι χρήστες μπορούν να περιγράψουν τις εφαρμογές και τις απαιτήσεις τους ως μοντέλα εφαρμογών (χρησιμοποιώντας την \selectlanguage{english}Cloud Application Modeling and Execution Language\selectlanguage{greek} ή \selectlanguage{english}CAMEL\selectlanguage{greek}). Η πλατφόρμα υποστηρίζει την συλλογή αποτελεσμάτων εκτελέσεων των εφαρμογών σε πολλαπλά υπολογιστικά νέφη και την αποθήκευσή τους σε ειδικά σχεδιασμένο αποθετήριο δεδομένων.

Η υλοποίηση της πλατφόρμας κοινωνικής δικτύωσης \selectlanguage{english}PaaSage\selectlanguage{greek} παρέχει στους χρήστες πληροφορίες που έχουν προέλθει από ιστορικά δεδομένα εκτελέσεων κατανεμημένων εφαρμογών, διευκολύνοντάς τους στην επιλογή εγκατάστασης (νέφη, εικονικές μηχανές, κλπ) με βάση σύνθετα κριτήρια, όπως η ανάλυση κόστους-αποτελεσματικότητας (\selectlanguage{english}cost effectiveness\selectlanguage{greek}).  Στην εργασία διερευνούνται και αξιολογούνται τεχνικές για την βελτίωση της κλιμακωσιμότητας της πλατφόρμας. Τέλος, για την καλύτερη καθοδήγηση των χρηστών που θέτουν τεχνικές ερωτήσεις στις βέλτιστες πιθανές απαντήσεις, αξιοποιούμε συστήματα κατηγοριοποίησης θεμάτων για την συσχέτιση ερωτήσεων των χρηστών με αποθηκευμένες ερωτήσεις και απαντήσεις (μερικές από τις οποίες περιλαμβάνουν αποτελέσματα από επερωτήσεις (\selectlanguage{english}queries\selectlanguage{greek}) επι ιστορικών δεδομένων αποτελεσμάτων εκτελέσεων εφαρμογών). Η υλοποίηση της πλατφόρμας είναι σε πιλοτική λειτουργία εντός του έργου \selectlanguage{english}PaaSage\selectlanguage{greek} από τον Μάρτιο του 2015.



\vfill

\end{titlepage}

\selectlanguage{english}
