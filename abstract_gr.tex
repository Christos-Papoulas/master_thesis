\thispagestyle{empty}
\selectlanguage{greek}
\begin{titlepage}
\begin{center}
{\bf\Large{Περίληψη}}\\
\end{center}

\indent  

Μια νέα τάση ανάμεσα στην ανάπτυξη και την λειτουργία των συστημάτων εφαρμογών που είναι γνωστή ως  \selectlanguage{english}DevOps\selectlanguage{greek} έχει γνωρίσει σημαντική ανάπτυξη τελευταία. Ένα τμήμα των μηχανικών \selectlanguage{english}DevOps\selectlanguage{greek}, οι οποίοι είναι ειδικοί στην προσαρμογή και εγκατάσταση των εφαρμογών σε περιβάλλοντα υπολογιστικού νέφους (επίσης γνωστοί ως \selectlanguage{english}cloud deployment specialists\selectlanguage{greek}), χρησιμοποιούν όλο και περισσότερο συγκεκριμένα εργαλεία για την εγκατάσταση των εφαρμογών τους όπως το \selectlanguage{english}Chef Supermarket\selectlanguage{greek} και το \selectlanguage{english}IBM Bluemix\selectlanguage{greek}.
Παρά την ύπαρξη αυτομάτων μηχανισμών, η κατάληξη σε καλή εγκατάσταση ακόμα απαιτεί συζήτηση με ειδικούς σε διαδικτυακά τεχνικά φόρουμ και κοινωνικά δίκτυα. 

Ανάμεσα στην κοινότητα των  \selectlanguage{english}DevOps\selectlanguage{greek}, η συζήτηση γύρω από την δομή των εφαρμογών και τα αποτελέσματα της εγκατάστασης σε υπολογιστικά νέφοι μπορεί να γίνει πιο εποικοδομητική γεφυρώνοντας τις τεχνολογίες των κοινωνικών δικτύων με την γνώση που υπάρχει στους χώρους δεδομένων που πηγάζουν από την παγκόσμια κοινότητα. Σε αυτήν την εργασία προτείνουμε μια πλατφόρμα κοινωνικού δικτύου (που παίρνει το όνομά της από το \selectlanguage{english}PaaSage EU project\selectlanguage{greek}) όπου οι ειδικοί στην εγκατάσταση σε περιβάλλοντα νέφους μπορούν να περιγράψουν τις εφαρμογές και τις απαιτήσεις τους ως μοντέλα εφαρμογών (χρησιμοποιώντας την \selectlanguage{english}Cloud Application Modeling and Execution Language\selectlanguage{greek} ή \selectlanguage{english}CAMEL\selectlanguage{greek}). Ακόμα μπορούν να συλλέξουν τα αποτελέσματα των εκτελέσεων των εφαρμογών σε πολλαπλά υπολογιστικά νέφη και να τα αποθηκεύσουν σε ένα χώρο δεδομένων που έχει δημιουργηθεί για αυτόν τον σκοπό. Τέλος, μπορούν να συζητήσουν με άλλους ειδικούς πάνω στα θέματα σχεδίασης και εγκατάστασης των εφαρμογών χρησιμοποιώντας αποτελέσματα εκτελέσεων.

Η υλοποίηση της πλατφόρμας κοινωνικής δικτύωσης του \selectlanguage{english}PaaSage\selectlanguage{greek} παρέχει στους χρήστες πληροφορίες που έχουν προέλθει από συλλογές εκτελέσεων κατανεμημένων εφαρμογών,  διευκολύνοντάς τους στην επιλογή της πλατφόρμας εγκατάστασης με βάση πληθώρα κριτηρίων (όπως η αποτελεσματικότητα κόστους). Η έρευνά μας εξερευνά και αξιολογεί διάφορες τεχνικές για την βελτίωση της κλιμακωσιμότητας της πλατφόρμας. Τέλος, για να κατευθύνουμε τους ειδικούς στην εγκατάσταση εφαρμογών στις βέλτιστες πιθανές απαντήσεις στης ερωτήσεις τους αξιοποιήσαμε εργαλεία κατηγοριοποίησης θεμάτων για να συσχετίσουμε τις ερωτήσεις των χρηστών με σχετικές ερωτήσεις και απαντήσεις (μερικές από τις οποίες μπορεί να περιλαμβάνουν αποτελέσματα από επερωτήσεις πάνω σε παλαιότερες εκτελέσεις εφαρμογών).


\vfill

\end{titlepage}

\selectlanguage{english}
