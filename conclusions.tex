\chapter{Conclusions and Future Work}
\label{sec:conclusions}
This thesis describes and evaluates a scalable social networking architecture for DevOps engineers, with a special focus on cloud deployment specialists. The scalability of the architecture relies on the provisioning of multiple social networking engine instances at the front end, and/or several memcached nodes at the back end of the system. Users of the social networking platform can benefit from the community knowledge and from the CAMEL repository of application models and executions, to improve the configuration, the deployment and the
optimization of distributed multi-cloud applications, tasks of major interest to cloud deployment specialists. 
%The design of our professional network applied best practices aiming to support the creation of a vigorous community, to allow users to retrieve timely and appropriate information and to carry out actions in the minimum number of steps.
Furthermore, we explored topic classification as a means to categorize community input and to better link it with existing content (past questions  and answers, and the results of past queries over historical execution data).

The user evaluations and pilot use of our platform within the PaaSage project has helped improve our implementation. As future work, we believe that extending the integration of our social networking platform with more information repositories such as GitHub, can provide additional benefits to the DevOps community.  A broader investigation of our topic classification system with a large user base is another promising avenue of future work.