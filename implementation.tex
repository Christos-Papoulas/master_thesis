
\chapter{Implementation}

This section describes the implementation of social network site
and how the system scales.

The system is composed by the following components, as shown in figure \ref{fig:system_architecture}: At the first layer lives (1) the Social Networking engine, which runs all PHP scripts and described in section \ref{sec:implementaion_of_social_netowrk}. At the second layer lives (2) the Memcached caching system, which described in section \ref{sec:memcache_implementation}. At the third layer lives (3) the Social Network MySQL database, and (4) the CDO server - client components and the CDO repository.

\begin{figure}[h]
	\caption{The overall architecture of Social Network.}
	\includegraphics[width=0.6\textwidth,natwidth=200,natheight=150]{./fig/system_architecture.pdf}
	\centering
	\label{fig:system_architecture}
\end{figure}

Achieving the scalability of the system, two system architectures are examined at two layers of the system: (1) We added more than one Social Network engine at the first layer of the system. In this implementation, in order to keep the file system in consistent mode we integraded Apache Zookeeper\cite{zookeeper_url}. (2) We added more than one memcached machines at the second layer in order to add more cpu capacity and improve the system response time.

\section{Implementation of Social Network}
\label{sec:implementaion_of_social_netowrk}
The social networking platform is implemented over the extensible Elgg social network framework\cite{elgg_url}.  Elgg is open source software written in PHP, uses MySQL for data persistence and supports jQuery~\cite{jquery_url} for client-side scripting.  The architecture of Elgg Social Network shown in figure ~\ref{fig:elgg_architecture}. The Model of the framework is structured around the following key concepts as shown in figure ~\ref{fig:elgg_entities}
\begin{itemize}
\item \emph{Entities}, classes capturing social networking concepts: users, communities, application models. Elgg Core comes with four basic objects: ElggObject, ElggUser, ElggGroup, ElggSite, ElggSession, ElggCache and a lot of other classes necessary for the proper engine operation.
\item \emph{Metadata} describing and extending entities (e.g., a response to a question, a review of an application model, etc.).
\item  \emph{Relationships} connecting two entities (e.g., user A is a friend of user B, user C is a contributor to an application model, etc.).
\item \emph{Annotations} are pieces of simple data attached to an entity that allow users to leave ratings, or other relevant feedback.
\end{itemize}
All Elgg objects inherit from ElggEntity, which provides the general attributes of an object. Elgg core comes with the following basic entities: ElggObject, ElggUser, ElggGroup, ElggSite, ElggSession, ElggCache, as well as other classes necessary for the operation of the engine.

\begin{figure}[h]
	\caption{The Elgg Engine Data model.}
	\includegraphics[width=0.6\textwidth,natwidth=200,natheight=150]{./fig/elgg_data_model.png}
	\centering
	\label{fig:elgg_entities}
\end{figure}

Elgg comprises a core system that can be extended through plugins (examples are the Cart system or the handling of Application Models). Plugins add new functionality, can customize aspects of the Elgg engine, or change the representation of pages.
A plugin can create new objects (e.g., ApplicationObject) characterized (through inheritance of ElggEntity) by a numeric globally unique identifier (GUID), owner GUID, Access ID. Access ID encodes permissions ensuring that when a page requests data it does not touch data the current user does not have permissions on. 

Figure\ref{fig:elgg_architecture} shows the model, view, and control parts of Elgg's architecture. In a typical scenario, a web client requests an HTML page (e.g., the description of an application model).  The request arrives at the \emph{Controller}, which confirms that the application exists and instructs \emph{Model} to increase the view counter on the application model object. The controller dispatches the request to the appropriate handler (e.g., application model, component handler, community handler) which then turns the request to the view system. View pulls the information about the application model and creates the HTML page returned to the web client.

\begin{figure}[h]
	\caption{Architecture of the Elgg Social Networking engine.}
	\includegraphics[width=0.6\textwidth,natwidth=200,natheight=150]{./fig/elgg_architecture.pdf}
	\centering
	\label{fig:elgg_architecture}
\end{figure}


The extensibility of Elgg can be established not by modifying the core system but by introducing new plug-ins which follow the MVC model. A new plug-in can create a new entity. Thus, each entity is characterized by a numeric Globally Unique Identifier and Access ID. The Access ID determines the permissions that other users have. Thus, when a page requests data, it never touches those data that the current user does not have permission to see. All plug-ins share a common structure of folders and PHP files, following the MVC model of figure~\ref{fig:elgg_architecture}. The hierarchy of a plug-in is shown in figure~\ref{fig:elgg_hierarchy}. Folder {\em actions} includes the actions applied on application models (delete, save, or search). The {\em views} folder contains the {\em php} forms applied on application models, {\em river} events (Elgg terminology for live feeds), and the application model editor. {\em Pages} overrides elements of core Elgg pages.  The {\em js} and {\em lib} folder provides javascript and {\em php} library functions. Finally, the {\em vendors} folders include third-party frameworks such as Twitter's bootstrap front-end.

\begin{figure}[h]
	\caption{The structure of the application description plug-in.}
	\includegraphics[width=0.6\textwidth,natwidth=200,natheight=150]{./fig/folder_hierarchy.jpg}
	\centering
	\label{fig:elgg_hierarchy}
\end{figure}

Social network relationships (friendship, group, ownership, etc.) are persisted in the Elgg back-end database. The execution history of deployments of application models and the description of those models is stored in the CAMEL information repository, which is implemented as an Eclipse CDO server. The exchange of information between Elgg and the CDO server is implemented over sockets.

\section{Memcache}
\label{sec:memcache_implementation}
This section describes the experience gained by using memcached\cite{memcache_url}. Memcached is an open source, high-performance, distributed memory object caching system. We choose memcached, because is a generic simple in-memory key-value store. It has a powerful API available for PHP. After memcached integration the system increase the responce time and performance.

Memcached stores all entities of Social Network, applications, components, users, group discussions and most important the executions of applications. Storing the executions of applications at Memcached the responce time of the system increased because the PHP modules do not need to go through the heavy CDO client but get directly the executions of applications from Memcached.

The apache jmeter\cite{jmeter_url} was used to measure the responce time of the system and the sysstat tool\cite{sysstat_url} was used to measure the cpu usage. Section \ref{sec:eval_memcache} 
 shows the performance results of this implementation.
