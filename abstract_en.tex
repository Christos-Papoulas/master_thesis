\thispagestyle{empty}
\begin{titlepage}
\begin{center}
{\bf\Large Abstract}\\
\end{center}

\indent A new discipline at the intersection of the development and operation of software systems known as DevOps has seen significant growth recently. A class of DevOps engineers that are experts in the configuration and deployment of applications on cloud environments (also known as cloud deployment specialists), increasingly use automated deployment and release-engineering tools like Chef Supermarket and IBM Bluemix to configure and deploy their applications.  Despite the advent of automated mechanisms, reasoning about good deployments still requires interaction with experts, often through discussions on online technical forums and social networks. 

Within the DevOps community, communication on application structure and cloud deployment tradeoffs could become more effective by bridging social networking technologies with knowledge present in global community-sourced information repositories.  In this work we propose a social networking platform (named after the PaaSage EU project) where cloud deployment specialists can express applications and their requirements as software models (using the Cloud Application Modeling and Execution Language or CAMEL), capture execution results from various multi-cloud platforms into a specifically-created information repository, and communicate with their peers on design and deployment issues, including deployments tradeoffs.

The implementation of the PaaSage social networking platform provides users with information mined from collected executions of distributed applications, facilitating their choice of deployment platform based on various criteria (such as cost effectiveness). Our investigation explores and evaluates several techniques to improve the scalability of the platform.  Finally, to better direct cloud deployment specialists to possible answers to their questions we leverage topic-classification tools to associate user questions with related questions and answers (some of which may contain the results of queries on the historical execution information). Our implementation is in pilot use within the PaaSage project since March 2015.

\vfill
\end{titlepage}

