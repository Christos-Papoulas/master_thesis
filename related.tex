\chapter{Related Work}
\label{sec:related}
In this chapter the related work of other professional networks is presented in Section~\ref{sec:networks_rel} and their caching architecture is described in Section~\ref{sec:caching_rel}. In addition, the Section~\ref{sec:config_manag} describes related tools and systems that the DevOps use in order to make their deployment faster. Finally, an overview and related work of topic classification is presented in Section~\ref{sec:topic_class}. 

\section{Professional and social networks} 
\label{sec:networks_rel}
This section reviews the related work on other professional social networking platforms and what they provide. Having the requirements of DevOps users as a priority, we compare those platforms with our proposed SNP, pointing their deficiencies and filling in the gaps with our innovation. 

Table~\ref{tab:related} summarizes
and categorizes the characteristics of the most important related approaches along the following dimensions
(depicted as columns of Table~\ref{tab:related}): 
\begin{itemize}
\item Which of the following key social features are supported by the platform: follow
users; news feeds; groups; Q\&A; personal messages;
\item Does the platform rely on one or more repositories
to store the following type of information: software code,
software models, configuration information, execution
histories; and whether these repositories are community sourced;
\item Does the social networking platform leverage
the repositories to provide users with specific suggestions
and hints; 
\item Does it support application deployment?
\end{itemize}

Information Technology (IT)~\cite{jorgenson1999information} professionals use a variety of online sources as aids in their daily tasks. Developers typically prefer community-moderated forums over vendor-moderated sites~\cite{evans-date}. Social networks focusing on software technology in particular provide developers with the opportunity to leverage the knowledge and expertise of their peers.
 
One of the most popular such platforms is GitHub~\cite{github_url}, a collaborative revision control platform for developers launched in April 2008, and arguably the largest code-hosting site in the world. 
GitHub provides social networking functionality such as feeds, followers, wikis and a social network graph that captures how developers work on versions of their repositories, which version is newest, etc.
Gitter~\cite{gitter} is a related service that facilitates discussions between members of GitHub communities by providing a long-term chat integrated with code and issues.
Sourceforge~\cite{sourceforge} was the first code-hosting platform offered to open-source projects. It was launched in 1999 and offered IT professionals the ability to develop, download, review, and publish open-source software. Sourceforge is similar to GitHub in its support for social features.
Other similar code-hosting platforms are Google Code~\cite{googlecode} and Microsoft CodePlex~\cite{codeplex}.
None of those platforms collect, analyze, or use information from executions of application deployments to improve the level of technical discussion between users or abstract code structure through modelling or enhance user interactions through the use of analytics over application execution histories.

StackOverflow~\cite{stackoverflow} advances on earlier community-driven Q\&A sites in which users ask and answer questions. Users can vote up or down questions and answers and earn \emph{reputation points} and \emph{badges} in return for their active participation. 
Although StackOverflow and GitHub address different aspects of software development (StackOverflow is not a code-hosting platform) there is a synergy and correlation between the two~\cite{stackgit}. The proposed social network platform extends StackOverflow through the use of social networking features that enable users interested in reasoning about application deployments to use and share knowledge drawn from analyses of information repositories.

\begin{table*}

\begin{threeparttable}
\begin{tabular}{c|c|c|c|c|c|c|c|c|c|cc}
\cline{2-10}
  &  \multicolumn{4}{c|}{User Interaction} & \multicolumn{5}{c|}{Repository} &  &  \\ 
\cline{2-12} 
  &
   
  \begin{tabular}[c]{@{}c@{}}\rot{Social features \tnote{a}} \end{tabular} & 
  \begin{tabular}[c]{@{}c@{}}\rot{Groups} \end{tabular} &
  \begin{tabular}[c]{@{}c@{}}\rot{Q \& A} \end{tabular} &
  \begin{tabular}[c]{@{}c@{}}\rot{Personal  messaging} \end{tabular} & 
  \begin{tabular}[c]{@{}c@{}}\rot{Software code}\end{tabular} & 
  \begin{tabular}[c]{@{}c@{}}\rot{Software models}\end{tabular} & 
  \begin{tabular}[c]{@{}c@{}}\rot{Software config}\end{tabular} & 
  \begin{tabular}[c]{@{}c@{}}\rot{Execution histories}\end{tabular} & 
  \begin{tabular}[c]{@{}c@{}}\rot{Crowd sourced}\end{tabular} & 
  \multicolumn{1}{c|}{\begin{tabular}[c]{@{}c@{}}Repo\\ assisted\\ hints \tnote{b} \end{tabular}} & 
  \multicolumn{1}{c|}{\begin{tabular}[c]{@{}c@{}}Application\\ deployment\end{tabular}} \\ 
\hline

\multicolumn{1}{|c|}{GitHub} & \cmark & \xmark & \xmark & \xmark & \cmark & \xmark & \xmark & \xmark & \cmark & \multicolumn{1}{c|}{\xmark} & \multicolumn{1}{c|}{\xmark} \\
\hline
\multicolumn{1}{|c|}{Sourceforge} & \cmark & \xmark & \cmark & \cmark & \cmark & \xmark & \xmark & \xmark & \cmark & \multicolumn{1}{c|}{\xmark} & \multicolumn{1}{c|}{\xmark} \\
\hline
\multicolumn{1}{|c|}{GoogleCode} & \xmark & \xmark & \cmark & \xmark & \cmark & \xmark & \xmark & \xmark & \cmark & \multicolumn{1}{c|}{\xmark} & \multicolumn{1}{c|}{\xmark} \\
\hline
\multicolumn{1}{|c|}{CodePlex} & \cmark & \xmark & \cmark & \cmark & \cmark & \xmark & \xmark & \xmark & \cmark & \multicolumn{1}{c|}{\xmark} & \multicolumn{1}{c|}{\xmark} \\
\hline
\multicolumn{1}{|c|}{StackOverflow} & \xmark & \xmark & \cmark & \xmark & \xmark & \xmark & \xmark & \xmark & \xmark & \multicolumn{1}{c|}{\xmark} & \multicolumn{1}{c|}{\xmark} \\ 
\hline
\multicolumn{1}{|c|}{BlueMix} & \xmark & \xmark & \cmark & \xmark & \xmark & \xmark & \cmark & \xmark & \xmark & \multicolumn{1}{c|}{\xmark} & \multicolumn{1}{c|}{\cmark} \\ 
\hline
\multicolumn{1}{|c|}{Chef Supermarket} & \xmark & \xmark & \xmark & \xmark & \xmark & \xmark & \cmark & \xmark & \cmark & \multicolumn{1}{c|}{\xmark} & \multicolumn{1}{c|}{\xmark} \\ 
\hline
\multicolumn{1}{|c|}{LinkedIn} & \cmark & \cmark & \cmark & \cmark & \xmark & \xmark & \xmark & \xmark & \xmark & \multicolumn{1}{c|}{\xmark} & \multicolumn{1}{c|}{\xmark} \\ 
\hline
\multicolumn{1}{|c|}{geeklist} & \cmark & \cmark & \xmark & \xmark & \xmark & \xmark & \xmark & \xmark & \cmark & \multicolumn{1}{c|}{ \xmark } & \multicolumn{1}{c|}{ \xmark } \\
\hline
\multicolumn{1}{|c|}{Snipplr} & \xmark & \xmark & \xmark & \xmark & \cmark & \xmark & \xmark & \xmark & \xmark  & \multicolumn{1}{c|}{\xmark} & \multicolumn{1}{c|}{\xmark} \\ 
\hline
\multicolumn{1}{|c|}{Masterbranch} & \xmark & \xmark & \xmark & \xmark & \cmark & \xmark & \xmark & \xmark & \xmark & \multicolumn{1}{c|}{\xmark} & \multicolumn{1}{c|}{\xmark} \\ 
\hline
\multicolumn{1}{|c|}{Dzone} & \cmark & \xmark & \xmark & \xmark & \cmark & \xmark & \xmark & \xmark & \xmark  & \multicolumn{1}{c|}{ \xmark} & \multicolumn{1}{c|}{\xmark } \\ 
\hline
\multicolumn{1}{|c|}{codeproject} & \cmark & \xmark & \cmark & \xmark & \cmark & \xmark & \xmark & \xmark & \cmark & \multicolumn{1}{c|}{\xmark} & \multicolumn{1}{c|}{\xmark} \\
\hline
\hline
\multicolumn{1}{|c|}{PaaSage SN} & \cmark & \cmark & \cmark & \cmark & \xmark & \cmark & \cmark & \cmark & \cmark & \multicolumn{1}{c|}{\cmark} & \multicolumn{1}{c|}{\cmark} \\ 
\hline

\end{tabular}

\begin{tablenotes}
      \small
       \item[a] Features: follow and news feed
      \item[b] User assistance based on data analysis of the repository
\end{tablenotes}
\end{threeparttable}
\caption{Feature comparison of other platforms}\label{tab:related}
\end{table*}

IBM's BlueMix~\cite{Bluemix-dev} is a key component of IBM's DevOps best practices~\cite{ibm-devops} for achieving rapid prototyping, automated deployment, and continuous testing of software. BlueMix encourages its users to ask their questions to StackOverflow but also includes a community forum~\cite{Bluemix-dev} with rating of answers contributes to eventually building a basic knowledge base, similar to traditional approaches such as StackOverflow.  The proposed social network platform system differs from BlueMix in its support for expressing applications as models (CloudML, CAMEL) and its use of two information repositories, the PaaSage repository of models and execution histories and Chef supermarket, and the use of analytics over past executions to enable users to reason about application deployments. A common feature between the proposed social network platform and BlueMix is support for deployment of distributed applications. 

Linkedin is widely adopted across a range of professional communities due to its robust set of social features (and to some extent due to its use of extensive analytics over collected information~\cite{sumbaly2013big}), LinkedIn provides no specific support for software engineering activities and thus more closely resembles traditional social networking platforms such as Facebook.

The lack of Social Networking features of github came to fill the Geeklist platform~\cite{geeklist_url}, where developers and IT companies can discover and share the work they have done, connect with other companies in a social network manner or join development communities. Another code hosting platform is Snipplr ~\cite{snipplr_url}, where developers can upload short code snippets but not full programs, in order to keep all of their frequently used code in one place that is accessible from any computer and any user. Masterbranch~\cite{masterbranch_url} is a new under development platform that allows collating and sharing of projects within a user's profile. This profile works similarly to LinkedIn and has an incentivisation scheme called DevScore, coupled with unlockable achievements that add a gamification element. Dzone~\cite{dzone_url} is essentially a link repository for developers allowing link sharing and incentivisation based on voting for the popular links. The Code project~\cite{codeproject_url} website and forum allow code-specific discussion and share relevant articles and news, contains blogs, newsletter and a questions and answers section.

The above systems can be further classified based on whether they use a repository to store software-related information (code, models, configuration, or execution histories) and whether this information is shared and raised through crowd sourcing~\cite{	howe2006rise}.  GitHub, GoogleCode, CodePlex, SourceForge, BlueMix, Chef supermarket, and our platform store at least one type of software-related information and all systems but BlueMix are raising shared content in their software-related repositories via crowd sourcing. Our social network platform is the only solution that analyzes information in its software-related repositories in order to provide users with assisting suggestions and hints. The platform targets the model of the application and the Software code of the application is not stored inside the social network platform. 

\section{Scalability in social networks}
\label{sec:caching_rel}
Arguably two of the largest existing networking platforms are Linkedin and Facebook. The caching technologies of these networks are of great interest for the way they are managing and storing vast amounts of data.

Linkedin, the largest professional network, stores hundreds of terabytes of data to Project Voldemort~\cite{sumbaly2012serving}, a key-value store, inspired by Amazon Dynamo~\cite{decandia2007dynamo}, another well-known key-value store. Linkedin stores to Voldemort pre-computed offline data. For example, it stores the results of data mining applications, such as ``People You May Know'' feature, that are running on hundreds of terabytes to make an estimation and are using Hadoop as the computational component of those estimations. 
Voldemort and Dynamo have the same following requirements: (1) a simple \emph{get/put} application interface (2) A \emph{replication} factor, the number of replicas for each key-value tuble, implemented using vector clock, (3) a \emph{required read} factor to succeed a get request and (4) a \emph{required write} factor to succeed a put request. 

Facebook, the largest social network, serves billions of requests per second using memcached~\cite{nishtala2013scaling}. In this magnitude of scale, Facebook has several pools of memcached servers (regional pools) around the globe. A request for a single page can produce hundred of requests to the back-end system. Memcached is used to store not only key-value from MySQL queries but also pre-computed results from sophisticated algorithms. 
In order to achieve a near real time communication experience to the end user, memcached servers have to be efficient, reducing latency to minimum. 

The research question in such systems is when a particular key will be invalidated. This problem occurs according to ~\cite{nishtala2013scaling} in two cases: (1) \emph{stale sets} and (2) \emph{thundering herds}. A stale set occurs when a web server sets a value to the memcached that does not reflect the real value of the database. Thundering herds occur when a specific key has a heavy read and write activity at the same time. Stale sets are resolved by an N-bit token, that is bound to a specific key and sent from the memcached to the web server that wants to update the key when a cache miss occurs. If a delete request is received, the request for updating this value from the client is rejected. The thundering herbs are solved by configuring the memcached servers to return an N-bit token only once every ten seconds per key.

The PaaSage Social Network, inspired by Facebook, integrates memcached in its back-end architecture and it is configured to properly interact with the caching application.


\section{Configuration management and deployment}
\label{sec:config_manag}
Another key component in a portfolio of DevOps tools is configuration management (CM)~\cite{lueninghoener2011getting}, the process of maintaining a detailed recording of software and hardware components in an infrastructure. An effective CM process provides significant benefits including reduced complexity through abstraction, greater flexibility, faster machine deployment, faster disaster recovery, etc. There are numerous configuration management tools from which a system administrator can choose, however the most widely known are: Bcfg2~\cite{Bcfg2}, CFEngine~\cite{CFEngine}, Chef~\cite{Chef_base} and Puppet~\cite{Puppet}. Each of these tools has its strengths and weaknesses~\cite{tsalolikhin2010summary},~\cite{Delaet2010a}. In a DevOps environment, a CM solution is often combined with provisioning and deployment tooling~\cite{ibm-devops}.

The social network platform uses Chef as a CM and deployment automation tool to support professional network users, and SNP integrates the Chef cookbooks in its platform.

Furthermore, a recent trend in DevOps software development is continuous integration (CI)~\cite{fowler2006continuous} and automated code deployment and testing off of online code repositories. Travis~\cite{travis_url} is a CI tool that automatically detects when a commit has been made and pushed to a GitHub repository, subsequently tries to build the project, deploy and run tests, and notify the user of the status. Another popular CI tool is Jenkins~\cite{jenkins_url}, an open-source software tool for testing and reporting on isolated code changes in real time. Similar to Travis, Jenkins enables developers to find and solve defects in their code rapidly and automates the testing of their builds. 

Although our social networking platform does not provide a complete CI solution because it is not a code hosting platform, but it automates the deployment of complex applications through a model-driven process. 

\section{Topic classification}
\label{sec:topic_class}
Another technology that has been frequently used by networking platforms and we have integrated to our SNP is Natural Language Processing (NLP)~\cite{manning1999foundations}. Specifically, we focus on social networking usage of NLP and how knowledge can be mined from repositories of Q\&A sites. 

NLP has been used to process Twitter's messages and come to some results according to the classifications. Twitter has a good pool of micro-blog text which is suitable for NLP because of the small text sentences that users are allowed to post. Those posts describe emotions, feelings, opinions or situations. So several techniques~\cite{pak2010twitter}~\cite{verma2011natural}~\cite{go2009twitter} have been introduced to process and classify twitter posts in several categories.

StackOverflow (SO) is not left without NLP, because it can be seen as a repository of Q\&A for programming questions. This means, that most of the questions and answers in SO contain some kind of a description at first and some code afterwards.    
An autoComment tool~\cite{wong2013autocomment} is proposed using NLP which maps the code from developer projects and locates the same code somewhere in the SO Q\&A, if it exists. If autoComment matches a segment of the developer's code with a code segment at SO, it performs NLP to the description of the code and inserts the modified description to the developer's code. 

A trend in Social Networking sites is the ability of users to ``tag'' their posts. Those tags describe the users' goals and interests. Tagging SO questions involves askers selecting appropriate keywords to broadly identify the domains
to which their questions are related. There also exist mechanisms by which other users can subscribe to tags, search via
tags, mark tags as favorites, etc. This users' classification of context is used by PaaSage social network platform as described in Section~\ref{sec:natural_implementation}. 

The social network platform uses those tags and the Natural Language Processing to answer the following research questions:
\begin{itemize}
\item Can the platform identify the similarity of a given question with other questions already posted in the system.
\item Can the platform map a question with a relevant query to the repository in order to provide the one who asks with an appropriate response.
\item Can the platform paraphrase the queries according to the user's arbitrary input in order to meet the previous objective. 
\end{itemize}
